%% Generated by Sphinx.
\def\sphinxdocclass{report}
\documentclass[letterpaper,10pt,english]{sphinxmanual}
\ifdefined\pdfpxdimen
   \let\sphinxpxdimen\pdfpxdimen\else\newdimen\sphinxpxdimen
\fi \sphinxpxdimen=.75bp\relax
\ifdefined\pdfimageresolution
    \pdfimageresolution= \numexpr \dimexpr1in\relax/\sphinxpxdimen\relax
\fi
%% let collapsible pdf bookmarks panel have high depth per default
\PassOptionsToPackage{bookmarksdepth=5}{hyperref}

\PassOptionsToPackage{warn}{textcomp}
\usepackage[utf8]{inputenc}
\ifdefined\DeclareUnicodeCharacter
% support both utf8 and utf8x syntaxes
  \ifdefined\DeclareUnicodeCharacterAsOptional
    \def\sphinxDUC#1{\DeclareUnicodeCharacter{"#1}}
  \else
    \let\sphinxDUC\DeclareUnicodeCharacter
  \fi
  \sphinxDUC{00A0}{\nobreakspace}
  \sphinxDUC{2500}{\sphinxunichar{2500}}
  \sphinxDUC{2502}{\sphinxunichar{2502}}
  \sphinxDUC{2514}{\sphinxunichar{2514}}
  \sphinxDUC{251C}{\sphinxunichar{251C}}
  \sphinxDUC{2572}{\textbackslash}
\fi
\usepackage{cmap}
\usepackage[T1]{fontenc}
\usepackage{amsmath,amssymb,amstext}
\usepackage{babel}



\usepackage{tgtermes}
\usepackage{tgheros}
\renewcommand{\ttdefault}{txtt}



\usepackage[Bjarne]{fncychap}
\usepackage{sphinx}

\fvset{fontsize=auto}
\usepackage{geometry}


% Include hyperref last.
\usepackage{hyperref}
% Fix anchor placement for figures with captions.
\usepackage{hypcap}% it must be loaded after hyperref.
% Set up styles of URL: it should be placed after hyperref.
\urlstyle{same}


\usepackage{sphinxmessages}




\title{MPM\_la}
\date{Oct 29, 2021}
\release{}
\author{unknown}
\newcommand{\sphinxlogo}{\vbox{}}
\renewcommand{\releasename}{}
\makeindex
\begin{document}

\pagestyle{empty}
\sphinxmaketitle
\pagestyle{plain}
\sphinxtableofcontents
\pagestyle{normal}
\phantomsection\label{\detokenize{index::doc}}



\chapter{A Gaussian Elimination routine}
\label{\detokenize{index:a-gaussian-elimination-routine}}
\sphinxAtStartPar
This package implements Gaussian elimination \sphinxstepexplicit %
\begin{footnote}[1]\phantomsection\label{\thesphinxscope.1}%
\sphinxAtStartFootnote
\sphinxurl{https://mathworld.wolfram.com/GaussianElimination.html}
%
\end{footnote} for \sphinxcode{\sphinxupquote{numpy.ndarray}} objects, along with hand\sphinxhyphen{}written matrix multiplication.

\sphinxAtStartPar
See {\hyperref[\detokenize{index:mpm_la.gauss}]{\sphinxcrossref{\sphinxcode{\sphinxupquote{mpm\_la.gauss()}}}}} and \sphinxcode{\sphinxupquote{mpm\_la.gauss.matmul()}} for more information.

\phantomsection\label{\detokenize{index:module-mpm_la}}\index{module@\spxentry{module}!mpm\_la@\spxentry{mpm\_la}}\index{mpm\_la@\spxentry{mpm\_la}!module@\spxentry{module}}\index{gauss() (in module mpm\_la)@\spxentry{gauss()}\spxextra{in module mpm\_la}}

\begin{fulllineitems}
\phantomsection\label{\detokenize{index:mpm_la.gauss}}\pysiglinewithargsret{\sphinxcode{\sphinxupquote{mpm\_la.}}\sphinxbfcode{\sphinxupquote{gauss}}}{\emph{\DUrole{n}{a}}, \emph{\DUrole{n}{b}}}{}
\sphinxAtStartPar
Given two matrices, \sphinxtitleref{a} and \sphinxtitleref{b}, with \sphinxtitleref{a} square, the determinant
of \sphinxtitleref{a} and a matrix \sphinxtitleref{x} such that a*x = b are returned.
If \sphinxtitleref{b} is the identity, then \sphinxtitleref{x} is the inverse of \sphinxtitleref{a}.
\begin{quote}\begin{description}
\item[{Parameters}] \leavevmode\begin{itemize}
\item {} 
\sphinxAtStartPar
\sphinxstyleliteralstrong{\sphinxupquote{a}} (\sphinxstyleliteralemphasis{\sphinxupquote{np.array}}\sphinxstyleliteralemphasis{\sphinxupquote{ or }}\sphinxstyleliteralemphasis{\sphinxupquote{list of lists}}) \textendash{} ‘n x n’ array

\item {} 
\sphinxAtStartPar
\sphinxstyleliteralstrong{\sphinxupquote{b}} (\sphinxstyleliteralemphasis{\sphinxupquote{np. array}}\sphinxstyleliteralemphasis{\sphinxupquote{ or }}\sphinxstyleliteralemphasis{\sphinxupquote{list of lists}}) \textendash{} ‘m x n’ array

\end{itemize}

\end{description}\end{quote}
\subsubsection*{Examples}

\begin{sphinxVerbatim}[commandchars=\\\{\}]
\PYG{g+gp}{\PYGZgt{}\PYGZgt{}\PYGZgt{} }\PYG{k+kn}{from} \PYG{n+nn}{mpm\PYGZus{}la} \PYG{k+kn}{import} \PYG{n}{gauss}
\PYG{g+gp}{\PYGZgt{}\PYGZgt{}\PYGZgt{} }\PYG{n}{a}\PYG{o}{=}\PYG{p}{[}\PYG{p}{[}\PYG{l+m+mi}{2}\PYG{p}{,}\PYG{l+m+mi}{0}\PYG{p}{,}\PYG{o}{\PYGZhy{}}\PYG{l+m+mi}{1}\PYG{p}{]}\PYG{p}{,}\PYG{p}{[}\PYG{l+m+mi}{0}\PYG{p}{,}\PYG{l+m+mi}{5}\PYG{p}{,}\PYG{l+m+mi}{6}\PYG{p}{]}\PYG{p}{,}\PYG{p}{[}\PYG{l+m+mi}{0}\PYG{p}{,}\PYG{o}{\PYGZhy{}}\PYG{l+m+mi}{1}\PYG{p}{,}\PYG{l+m+mi}{1}\PYG{p}{]}\PYG{p}{]}
\PYG{g+gp}{\PYGZgt{}\PYGZgt{}\PYGZgt{} }\PYG{n}{b}\PYG{o}{=}\PYG{p}{[}\PYG{p}{[}\PYG{l+m+mi}{2}\PYG{p}{]}\PYG{p}{,}\PYG{p}{[}\PYG{l+m+mi}{1}\PYG{p}{]}\PYG{p}{,}\PYG{p}{[}\PYG{l+m+mi}{2}\PYG{p}{]}\PYG{p}{]}
\PYG{g+gp}{\PYGZgt{}\PYGZgt{}\PYGZgt{} }\PYG{n}{det}\PYG{p}{,}\PYG{n}{x}\PYG{o}{=}\PYG{n}{gauss}\PYG{p}{(}\PYG{n}{a}\PYG{p}{,}\PYG{n}{b}\PYG{p}{)}
\PYG{g+gp}{\PYGZgt{}\PYGZgt{}\PYGZgt{} }\PYG{n}{det}
\PYG{g+go}{22.0}
\PYG{g+gp}{\PYGZgt{}\PYGZgt{}\PYGZgt{} }\PYG{n}{x}
\PYG{g+go}{[[1.5], [\PYGZhy{}1.0], [1.0]]}
\PYG{g+gp}{\PYGZgt{}\PYGZgt{}\PYGZgt{} }\PYG{k+kn}{from} \PYG{n+nn}{mpm\PYGZus{}la} \PYG{k+kn}{import} \PYG{n}{gauss}
\PYG{g+gp}{\PYGZgt{}\PYGZgt{}\PYGZgt{} }\PYG{n}{A}\PYG{o}{=}\PYG{p}{[}\PYG{p}{[}\PYG{l+m+mi}{1}\PYG{p}{,}\PYG{l+m+mi}{0}\PYG{p}{,}\PYG{o}{\PYGZhy{}}\PYG{l+m+mi}{1}\PYG{p}{]}\PYG{p}{,}\PYG{p}{[}\PYG{o}{\PYGZhy{}}\PYG{l+m+mi}{2}\PYG{p}{,}\PYG{l+m+mi}{3}\PYG{p}{,}\PYG{l+m+mi}{0}\PYG{p}{]}\PYG{p}{,}\PYG{p}{[}\PYG{l+m+mi}{1}\PYG{p}{,}\PYG{o}{\PYGZhy{}}\PYG{l+m+mi}{3}\PYG{p}{,}\PYG{l+m+mi}{2}\PYG{p}{]}\PYG{p}{]}
\PYG{g+gp}{\PYGZgt{}\PYGZgt{}\PYGZgt{} }\PYG{n}{I}\PYG{o}{=}\PYG{p}{[}\PYG{p}{[}\PYG{l+m+mi}{1}\PYG{p}{,}\PYG{l+m+mi}{0}\PYG{p}{,}\PYG{l+m+mi}{0}\PYG{p}{]}\PYG{p}{,}\PYG{p}{[}\PYG{l+m+mi}{0}\PYG{p}{,}\PYG{l+m+mi}{1}\PYG{p}{,}\PYG{l+m+mi}{0}\PYG{p}{]}\PYG{p}{,}\PYG{p}{[}\PYG{l+m+mi}{0}\PYG{p}{,}\PYG{l+m+mi}{0}\PYG{p}{,}\PYG{l+m+mi}{1}\PYG{p}{]}\PYG{p}{]}
\PYG{g+gp}{\PYGZgt{}\PYGZgt{}\PYGZgt{} }\PYG{n}{Det}\PYG{p}{,}\PYG{n}{Ainv}\PYG{o}{=}\PYG{n}{gauss}\PYG{p}{(}\PYG{n}{A}\PYG{p}{,} \PYG{n}{I}\PYG{p}{)}
\PYG{g+gp}{\PYGZgt{}\PYGZgt{}\PYGZgt{} }\PYG{n}{Det}
\PYG{g+go}{3.0}
\end{sphinxVerbatim}
\subsubsection*{Notes}

\sphinxAtStartPar
See \sphinxurl{https://en.wikipedia.org/wiki/Gaussian\_elimination}         for further details.

\end{fulllineitems}



\begin{fulllineitems}
\pysiglinewithargsret{\sphinxcode{\sphinxupquote{mpm\_la.gauss.}}\sphinxbfcode{\sphinxupquote{matmul}}}{\emph{\DUrole{n}{a}}, \emph{\DUrole{n}{b}}}{}
\sphinxAtStartPar
Given two matrices, \sphinxtitleref{a} and \sphinxtitleref{b}. First, determine the shape         of the result matrix after multiplication
according to the shapes of the matrices a and b, and generate         a zero matrix of the corresponding shape.
Next, complete the matrix multiplication and return the result matrix.
\begin{quote}\begin{description}
\item[{Parameters}] \leavevmode\begin{itemize}
\item {} 
\sphinxAtStartPar
\sphinxstyleliteralstrong{\sphinxupquote{a}} (\sphinxstyleliteralemphasis{\sphinxupquote{np.array}}\sphinxstyleliteralemphasis{\sphinxupquote{ or }}\sphinxstyleliteralemphasis{\sphinxupquote{list of lists}}) \textendash{} 

\item {} 
\sphinxAtStartPar
\sphinxstyleliteralstrong{\sphinxupquote{array}} (\sphinxstyleliteralemphasis{\sphinxupquote{\textquotesingle{}m x n\textquotesingle{}}}) \textendash{} 

\item {} 
\sphinxAtStartPar
\sphinxstyleliteralstrong{\sphinxupquote{b}} (\sphinxstyleliteralemphasis{\sphinxupquote{np. array}}\sphinxstyleliteralemphasis{\sphinxupquote{ or }}\sphinxstyleliteralemphasis{\sphinxupquote{list of lists}}) \textendash{} 

\item {} 
\sphinxAtStartPar
\sphinxstyleliteralstrong{\sphinxupquote{array}} \textendash{} 

\end{itemize}

\end{description}\end{quote}
\subsubsection*{Examples}

\begin{sphinxVerbatim}[commandchars=\\\{\}]
\PYG{g+gp}{\PYGZgt{}\PYGZgt{}\PYGZgt{} }\PYG{k+kn}{from} \PYG{n+nn}{mpm\PYGZus{}la} \PYG{k+kn}{import} \PYG{n}{matmul}
\PYG{g+gp}{\PYGZgt{}\PYGZgt{}\PYGZgt{} }\PYG{n}{a}\PYG{o}{=}\PYG{p}{[}\PYG{p}{[}\PYG{l+m+mi}{1}\PYG{p}{,}\PYG{l+m+mi}{2}\PYG{p}{]}\PYG{p}{,}\PYG{p}{[}\PYG{l+m+mi}{3}\PYG{p}{,}\PYG{l+m+mi}{4}\PYG{p}{]}\PYG{p}{]}
\PYG{g+gp}{\PYGZgt{}\PYGZgt{}\PYGZgt{} }\PYG{n}{b}\PYG{o}{=}\PYG{p}{[}\PYG{p}{[}\PYG{l+m+mi}{5}\PYG{p}{]}\PYG{p}{,}\PYG{p}{[}\PYG{l+m+mi}{6}\PYG{p}{]}\PYG{p}{]}
\PYG{g+gp}{\PYGZgt{}\PYGZgt{}\PYGZgt{} }\PYG{n}{res\PYGZus{}mul}\PYG{o}{=}\PYG{n}{matmul}\PYG{p}{(}\PYG{n}{a}\PYG{p}{,}\PYG{n}{b}\PYG{p}{)}
\PYG{g+gp}{\PYGZgt{}\PYGZgt{}\PYGZgt{} }\PYG{n}{res\PYGZus{}mul}
\PYG{g+go}{[[17], [39]]}
\end{sphinxVerbatim}

\begin{sphinxVerbatim}[commandchars=\\\{\}]
\PYG{g+gp}{\PYGZgt{}\PYGZgt{}\PYGZgt{} }\PYG{k+kn}{from} \PYG{n+nn}{mpm\PYGZus{}la} \PYG{k+kn}{import} \PYG{n}{matmul}
\PYG{g+gp}{\PYGZgt{}\PYGZgt{}\PYGZgt{} }\PYG{n}{a}\PYG{o}{=}\PYG{p}{[}\PYG{p}{[}\PYG{l+m+mi}{1}\PYG{p}{,}\PYG{l+m+mi}{2}\PYG{p}{]}\PYG{p}{,}\PYG{p}{[}\PYG{l+m+mi}{3}\PYG{p}{,}\PYG{l+m+mi}{4}\PYG{p}{]}\PYG{p}{]}
\PYG{g+gp}{\PYGZgt{}\PYGZgt{}\PYGZgt{} }\PYG{n}{b}\PYG{o}{=}\PYG{p}{[}\PYG{p}{[}\PYG{l+m+mi}{5}\PYG{p}{,}\PYG{l+m+mi}{1}\PYG{p}{]}\PYG{p}{,}\PYG{p}{[}\PYG{l+m+mi}{6}\PYG{p}{,}\PYG{l+m+mi}{2}\PYG{p}{]}\PYG{p}{]}
\PYG{g+gp}{\PYGZgt{}\PYGZgt{}\PYGZgt{} }\PYG{n}{mul}\PYG{o}{=}\PYG{n}{matmul}\PYG{p}{(}\PYG{n}{a}\PYG{p}{,}\PYG{n}{b}\PYG{p}{)}
\PYG{g+gp}{\PYGZgt{}\PYGZgt{}\PYGZgt{} }\PYG{n}{mul}
\PYG{g+go}{[[17, 5], [39, 11]]}
\end{sphinxVerbatim}

\end{fulllineitems}



\begin{fulllineitems}
\pysiglinewithargsret{\sphinxcode{\sphinxupquote{mpm\_la.gauss.}}\sphinxbfcode{\sphinxupquote{zeromat}}}{\emph{\DUrole{n}{p}}, \emph{\DUrole{n}{q}}}{}
\sphinxAtStartPar
Given two integers, \sphinxtitleref{p} and \sphinxtitleref{q}. \sphinxtitleref{p} is the         number of rows in the first matrix,
and \sphinxtitleref{q} is the number of columns in the second matrix.         The function will return a matrix with all zero values.
The shape of the returned matrix is the same as the shape         as a result of multiplying two matrices.
\begin{quote}\begin{description}
\item[{Parameters}] \leavevmode\begin{itemize}
\item {} 
\sphinxAtStartPar
\sphinxstyleliteralstrong{\sphinxupquote{p}} (\sphinxstyleliteralemphasis{\sphinxupquote{Integer}}) \textendash{} 

\item {} 
\sphinxAtStartPar
\sphinxstyleliteralstrong{\sphinxupquote{q}} (\sphinxstyleliteralemphasis{\sphinxupquote{Integer}}) \textendash{} 

\end{itemize}

\end{description}\end{quote}
\subsubsection*{Examples}

\begin{sphinxVerbatim}[commandchars=\\\{\}]
\PYG{g+gp}{\PYGZgt{}\PYGZgt{}\PYGZgt{} }\PYG{k+kn}{from} \PYG{n+nn}{mpm\PYGZus{}la} \PYG{k+kn}{import} \PYG{n}{zeromat}
\PYG{g+gp}{\PYGZgt{}\PYGZgt{}\PYGZgt{} }\PYG{n}{p} \PYG{o}{=} \PYG{l+m+mi}{3}
\PYG{g+gp}{\PYGZgt{}\PYGZgt{}\PYGZgt{} }\PYG{n}{q} \PYG{o}{=} \PYG{l+m+mi}{1}
\PYG{g+gp}{\PYGZgt{}\PYGZgt{}\PYGZgt{} }\PYG{n}{res\PYGZus{}zero} \PYG{o}{=} \PYG{n}{zeromat}\PYG{p}{(}\PYG{n}{p}\PYG{p}{,} \PYG{n}{q}\PYG{p}{)}
\PYG{g+gp}{\PYGZgt{}\PYGZgt{}\PYGZgt{} }\PYG{n}{res\PYGZus{}zero}
\PYG{g+go}{[[0], [0], [0]]}
\end{sphinxVerbatim}

\begin{sphinxVerbatim}[commandchars=\\\{\}]
\PYG{g+gp}{\PYGZgt{}\PYGZgt{}\PYGZgt{} }\PYG{k+kn}{from} \PYG{n+nn}{mpm\PYGZus{}la} \PYG{k+kn}{import} \PYG{n}{zeromat}
\PYG{g+gp}{\PYGZgt{}\PYGZgt{}\PYGZgt{} }\PYG{n}{p} \PYG{o}{=} \PYG{l+m+mi}{4}
\PYG{g+gp}{\PYGZgt{}\PYGZgt{}\PYGZgt{} }\PYG{n}{q} \PYG{o}{=} \PYG{l+m+mi}{4}
\PYG{g+gp}{\PYGZgt{}\PYGZgt{}\PYGZgt{} }\PYG{n}{res\PYGZus{}zero} \PYG{o}{=} \PYG{n}{zeromat}\PYG{p}{(}\PYG{n}{p}\PYG{p}{,} \PYG{n}{q}\PYG{p}{)}
\PYG{g+gp}{\PYGZgt{}\PYGZgt{}\PYGZgt{} }\PYG{n}{res\PYGZus{}zero}
\PYG{g+go}{[[0, 0, 0, 0], [0, 0, 0, 0], [0, 0, 0, 0], [0, 0, 0, 0]]}
\end{sphinxVerbatim}

\end{fulllineitems}



\chapter{Another Algorithm to Compute the Determinant}
\label{\detokenize{index:another-algorithm-to-compute-the-determinant}}
\sphinxAtStartPar
This package also implements another algorithm for \sphinxcode{\sphinxupquote{numpy.ndarray}} objects, to compute the determinant of a single square matrix.

\sphinxAtStartPar
See \sphinxcode{\sphinxupquote{mpm\_la.det()}} for more information.

\phantomsection\label{\detokenize{index:module-0}}\index{module@\spxentry{module}!mpm\_la@\spxentry{mpm\_la}}\index{mpm\_la@\spxentry{mpm\_la}!module@\spxentry{module}}

\begin{fulllineitems}
\pysiglinewithargsret{\sphinxcode{\sphinxupquote{mpm\_la.det.}}\sphinxbfcode{\sphinxupquote{det}}}{\emph{\DUrole{n}{mat}}}{}
\sphinxAtStartPar
Given one matrix, \sphinxtitleref{mat}, the determinant of \sphinxtitleref{mat}
will be returned.
\begin{quote}\begin{description}
\item[{Parameters}] \leavevmode
\sphinxAtStartPar
\sphinxstyleliteralstrong{\sphinxupquote{mat}} (\sphinxstyleliteralemphasis{\sphinxupquote{np.array}}\sphinxstyleliteralemphasis{\sphinxupquote{ or }}\sphinxstyleliteralemphasis{\sphinxupquote{list of lists}}) \textendash{} ‘n x n’ array

\end{description}\end{quote}

\end{fulllineitems}

\subsubsection*{References}


\renewcommand{\indexname}{Python Module Index}
\begin{sphinxtheindex}
\let\bigletter\sphinxstyleindexlettergroup
\bigletter{m}
\item\relax\sphinxstyleindexentry{mpm\_la}\sphinxstyleindexpageref{index:\detokenize{module-0}}
\end{sphinxtheindex}

\renewcommand{\indexname}{Index}
\printindex
\end{document}